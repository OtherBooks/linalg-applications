\achapter{41}{Complex Numbers} \label{sec:complex_numbers}

\vspace*{-17 pt}
\framebox{\hspace*{3 pt}
\parbox{4.7 in}{\begin{fqs}
\item What is a complex number?
\item How is the sum and product of two complex numbers defined?
\item How do we find the multiplicative inverse of a nonzero complex number?
\item What general structure does the set of complex numbers have?
\end{fqs}} \hspace*{3 pt}}

\vspace*{13 pt}

\csection{Complex Numbers}
Complex numbers are usually introduced as a tool to solve the quadratic equation $x^2+1 = 0$. However, that is not how complex numbers first came to light. The story actually involves solutions to the general cubic equation. The interested reader could consult Chapter 6 of William Dunham's excellent book \emph{Journey Through Genius}. In this appendix we touch on the basics of complex numbers to provide enough context for the section on complex eigenvalues. 

A complex number is defined by a pair of real numbers - the \emph{real part} of the complex number and the \emph{imaginary part} of the complex number. 

\begin{definition} A \textbf{complex number}\index{complex number} is a number of the form
\[a+bi\]
where $a$ and $b$ are real numbers and $i^2 = -1$. 
\end{definition}
The number $a$ is the real part\index{complex number!real part} of the complex number and the number $b$ is the imaginary part\index{complex number!imaginary part}. We often write
\[z = a+bi\]
for a complex number $z$, $\text{Re}(z)$ for the real part of $z$ and $\text{Im}(z)$ for the imaginary part of $z$. That is, if $z = a+bi$ with $a$ and $b$ real numbers, then $\text{Re}(z) = a$ and $\text{Im}(z) = b$. We say that two complex numbers $a+bi$ and $c+di$ are equal if $a=c$ and $b=d$. 

There is an arithmetic of complex numbers that is determined by an addition and multiplication of complex numbers. Adding complex numbers is natural:
\[(a+bi) + (c+di) = (a+c) + (b+d)i.\]
That is, to add two complex numbers we add their real parts together and their imaginary parts together. 

\begin{activity} Multiplication of complex numbers is is also done in a natural way. 
\ba
\item By expanding the product as usual, treating $i$ as we would any real number, and exploiting the fact that $i^2 = -1$, explain why we define the product of complex numbers $a+bi$ and $c+di$ as 
\[(a+bi)(c+di) = (ac-bd) + (bc+ad)i.\]

\item Use the definitions of addition and multiplication to write each of the sums or products as a complex number in the form $a+bi$.
	\begin{enumerate}[i.]
	\item $(2+3i) + (7-4i)$

	\item $(4-2i)(3+i)$

	\item $(2+i)i - (3+4i)$

	\end{enumerate}
\ea

\end{activity}


It isn't difficult to show that the set of complex numbers, which we denote by $\C$, satisfies many useful and familiar properties. 

\begin{activity} \label{act:complex_field} Show that $\C$ has the same structure as $\R$. That is, show that for all $u$, $w$, and $z$ in $\C$, the following properties are satisfied.
\ba
	\item $w+z \in \C$ and $wz \in \C$ 
	\item $w+z=z+w$ and $wz=zw$ 
	\item $(w+z) + u = w + (z + u)$ and $(wz)u = w(zu)$ 
	\item There is an element $0$ in $\C$ such that $z+ 0 = z$ 
	\item There is an element $1$ in $\C$ such that $(1)z = z$ 
	\item There is an element $-z$ in $\C$ such that $z+(-z) = 0$
	\item If $z \neq 0$, there is an element $\frac{1}{z}$ in $\C$ such that $z\left(\frac{1}{z}\right) = 1$
	\item $u (w + z) = (u w) + (u z)$ 
	\ea

\end{activity}

The result of Activity \ref{act:complex_field} is that, just like $\R$, the set $\C$ is a field. If we wanted to, we could define vector spaces over $\C$ just like we did over $\R$. The same results hold. 

\csection{Conjugates and Modulus}

We can draw pictures of complex numbers in the plane. We let the $x$-axis be the real axis for a complex number and the $y$-axis the imaginary axis. That is, if $z=a+bi$ we can think of $z$ as a directed line segment from the origin to the point $(a,b)$, where the terminal point of the segment is $a$ units from the imaginary axis and $b$ units from the real axis. For example, the complex numbers $3+4i$ and $-8+3i$ are shown in Figure \ref{F:complex_numbers}.
 \begin{figure}
\begin{center}
\resizebox{!}{1.5in}{\includegraphics{A_complex_numbers}}
\caption{\scriptsize Two complex numbers.}
\label{F:complex_numbers}
\end{center}
\end{figure}

We can also think of the complex number $z = a+bi$ as the vector $[a \ b]^{\tr}$. In this way, the set $\C$ is a two-dimensional vector space over $\R$ with basis $\{1, i\}$. Each of these complex numbers has a length that we call the \emph{norm} or \emph{modulus} of the complex number. We denote the norm of a complex number $a+bi$ as $|a+bi|$. The distance formula or the Pythagorean theorem show that
\[|a+bi| = \sqrt{a^2+b^2}.\]

Note that
\[a^2+b^2 = a^2-b^2i^2 = (a+bi)(a-bi)\]
so the norm of the complex number $a+bi$ can also be viewed as a square root of the product of $a+bi$ with $a-bi$. The number $a-bi$ is called the \emph{complex conjugate}\index{complex conjugate} of $a+bi$. If we let $z = a+bi$, we denote the complex conjugate of $z$ as $\overline{z}$. So $\overline{a+bi} = a-bi$.


\begin{activity} Let $w = 2+3i$ and $z = -1+5i$.
\ba
\item Find $\overline{w}$ and $\overline{z}$.

\item Compute $|w|$ and $|z|$.

\item Compute $w\overline{w}$ and $z \overline{z}$.


\item Let $z$ be an arbitrary complex number. There is a relationship between $|z|$, $z$, and $\overline{z}$. Find and verify this relationship.

\item What is $\overline{z}$ if $z \in \R$?

\ea

\end{activity}

\csection{Complex Vectors}

A vector can have real and imaginary parts, too. For example, the vector $\vv = \left[ \begin{array}{c} 1+i \\ 2-i \end{array} \right]$ can be written as 
\[\vv = \left[ \begin{array}{c} 1+i \\ 2-i \end{array} \right] = \left[ \begin{array}{c} 1 \\ 2 \end{array} \right] + \left[ \begin{array}{r} i \\ -i \end{array} \right] = \left[ \begin{array}{c} 1 \\ 2 \end{array} \right] + i\left[ \begin{array}{r} 1 \\ -1 \end{array} \right].\]
The vector $\text{Re}(\vv) = \left[ \begin{array}{c} 1 \\ 2 \end{array} \right]$ is the real part of $\vv$ and the vector $\text{Im}(z) = \left[ \begin{array}{r} 1 \\ -1 \end{array} \right]$ is the imaginary part of $\vv$. In this way any vector $\vv$ with complex entries can be written in the form 
\[\vv = \vx + i \vy,\]
where $\vx = \text{Re}(z)$ and $\vy = \text{Im}(z)$ are vectors with real entries. The conjugate of the vector $\vv = \vx + i \vy$ is the vector $\overline{\vv} = \vx - i \vy$. We can do the same with matrices with complex entries. 

There are several properties of complex conjugates that can be useful. Suppose $r$ is a complex number, $\vv$ is a vector with possibly complex entries, and $A$ and $B$ are matrices with possibly complex entries. Assume all products that are listed are defined. Then
\begin{enumerate}
\item $\overline{r\vv} = \overline{r} \ \overline{\vv}$ 
\item $\overline{rA} = \overline{r} \ \overline{A}$
\item $\overline{A\vv} = \overline{A} \ \overline{\vv}$ 
\item $\overline{AB} = \overline{A} \ \overline{B}$ 
\end{enumerate}

These properties can be verified using complex arithmetic. For example, to verify the first property, let $r = a+ib$ be a complex number and let $\vv = \vx + i \vy$ be a complex vector. The operations on complex numbers give us
\begin{align*}
\overline{r\vv} &= \overline{(a\vx-b\vy) + i(a\vy + b\vx)} \\
	&= (a\vx-b\vy) - i(a\vy + b\vx) \\
	&= (a-ib)(\vx - i\vy) \\
	&= \overline{r} \overline{\vv}.
\end{align*}
	
Verification of the remaining properties is left to the reader. 






