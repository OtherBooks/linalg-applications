\achapter{5}{The Matrix-Vector Form of a Linear System} \label{chap:matrix_vector}

\vspace*{-17 pt}
\framebox{
\parbox{\dimexpr\linewidth-3\fboxsep-3\fboxrule}
{\begin{fqs}
\item How and when is the matrix-vector product $A \vx$ defined? 
\item How can a system of linear equations be written in matrix-vector form? 
\item How can we tell if the system $A \vx = \vb$ is consistent for a given vector $\vb$?
\item How can we tell if the system $A \vx = \vb$ is consistent for every vector $\vb$?
\item What is a homogeneous system? What can we say about the solution set to a homogeneous system? 
\item What must be true about pivots in the coefficient matrix $A$ in order for the homogeneous system $A \vx = \vzero$ to have a unique solution? 
\item How are the solutions to the nonhomogeneous system $A \vx = \vb$ related to the solutions of the corresponding homogeneous system $A \vx = \vzero$?
\end{fqs}}}% \hspace*{3 pt}}

\vspace*{13 pt}

\csection{Application: Modeling an Economy}
\label{sec:appl_model_econ}

An economy is a very complex system. An economy is not a well-defined object, there are many factors that influence an economy, and it is often unclear how the factors influence each other. Mathematical modeling plays an important role in attempting to understand an economy.

In 1941 Wassily Leontief developed the first empirical model of a national economy. Around 1949 Leontief used data from the U.S. Bureau of Labor Statistics to divide the U.S. economy into 500 sectors. He then set up linear equations for each sector. This system was too large for the computers at the time to solve, so he then aggregated the information into 42 sectors. The Harvard Mark II computer was used to solve this system, one of the first significant uses of computers for mathematical modeling. Leontief won the 1973 Nobel Prize in economics for his work. 

With such large models (Leontief's models are called \emph{input-output} models) it is important to find a shorthand way to represent the resulting systems.  In this section we will see how to represent any size system of linear equations in a very convenient way. Later, we will analyze a small economy using input-output models. 

\csection{Introduction}
\label{sec:mv_form_intro}

There is another useful way to represent a system of linear equations using a matrix-vector product that we investigate in this section. To understand how this product comes about, recall that we can represent the linear system 
\begin{alignat*}{5}
x_1 		&{}+{} 	&{4}x_2 	&{}+{}	&{2}x_3 	&{}+{}	&{4}x_4	&{}={}	&1 \\
{2}x_1 	&{}-{} 	&{}x_2 		&{}-{}	&{5}x_3 	&{}-{}	&{}x_4	&{}={}	&2 \\
{3}x_1 	&{}+{} 	&{7}x_2 	&{}+{}	&{}x_3 		&{}+{}	&{7}x_4	&{}={}	&3 \\
\end{alignat*}
as a vector equation as
\begin{equation} \label{eq:PA1e_1}
x_1\left[ \begin{array}{c} 1\\2\\3 \end{array} \right] + x_2\left[ \begin{array}{r} 4\\-1\\7 \end{array} \right] + x_3\left[ \begin{array}{r} 2\\-5\\1 \end{array} \right] + x_4\left[ \begin{array}{r} 4\\-1\\7 \end{array} \right]= \left[ \begin{array}{c} 1\\2\\3 \end{array} \right].
\end{equation}
We can view the left hand side of Equation ( \ref{eq:PA1e_1}) as a \textbf{matrix-vector product}. Specifically, if $A = \left[ \begin{array}{crrr} 1&4&2&4 \\ 2&-1&-5&-1 \\ 3&7&1&7 \end{array} \right]$ and $\vx = \left[ \begin{array}{c} x_1 \\x_2 \\x_3 \\x_4 \end{array} \right]$, then we define the \emph{matrix-vector product} $A\vx$ as the left hand side Equation \eqref{eq:PA1e_1}. So the matrix-vector product $A\vx$ is the linear combination of the columns of $A$ with weights from the vector $\vx$ in order. 

With this definition, the vector equation in \eqref{eq:PA1e_1} can be expressed as a matrix-vector equation as
\[\left[ \begin{array}{crrr} 1&4&2&4 \\ 2&-1&-5&-1 \\ 3&7&1&7 \end{array} \right] \left[ \begin{array}{c} x_1 \\x_2 \\x_3\\x_4 \end{array} \right] = \left[ \begin{array}{c} 1\\2\\3 \end{array} \right] \,.\]
We call this representation the \textbf{matrix-vector form} of the system. Note that the matrix $A$ in this expression is the same as the coefficient matrix that appears in the augmented matrix representation of the system.

We can use the above definition of the matrix-vector product as a linear combination with any matrix and any vector, as long as it is meaningful to use the entries in the vector as weights for the columns of the matrix. For example, for $A=\left[ \begin{array}{cc} 1&2\\3&1\\1&1 \end{array} \right]$ and $\vv = \left[ \begin{array}{c} 3\\4 \end{array} \right]$, then we can define $A\vv$ to be the linear combination of the columns of $A$ with weights 3 and 4:
\[A\vv= 3 \left[ \begin{array}{c} 1\\3\\1 \end{array} \right] + 4 \left[ \begin{array}{c} 2\\1\\1 \end{array} \right] = \left[ \begin{array}{c} 11\\13\\7 \end{array} \right] \, .\]
However, note that if $\vv$ had three entries, this definition would not make sense since we do not have three columns in $A$. In those cases, we say $A\vv$ is not defined. We will later see that this definition can be generalized to matrix-matrix products, by treating the vector as a special case of a matrix with one column.

\begin{pa} \label{pa:1_e} ~
\be

\item Write the vector equation
\[x_1 \left[ \begin{array}{c} 1 \\ 1 \\ 2 \end{array} \right] +x_2 \left[ \begin{array}{c}  1\\ 2\\ 3 \end{array} \right] + x_3 \left[ \begin{array}{r} -1 \\ 2 \\ -3 \end{array} \right] = \left[ \begin{array}{c} 4 \\ 3\\ 11 \end{array} \right] \]
in matrix-vector form. Note that this is the vector equation whose augmented matrix representation was given in Problem 2 in Preview Activity \ref{pa:1_b}. Compare your matrix $A$ and the right hand side vector to the augmented matrix. Do not solve the system.

\item Given the matrix-vector equation
\[ \left[ \begin{array}{crr} 1 & 2 & -1 \\ 0&1&2\\1&-2&-3 \end{array} \right] \vx = \left[\begin{array}{r} -3\\3\\-7 \end{array}\right] \]
represent the system corresponding to this equation. Note that this should correspond to the system (or an equivalent system where an equation might be multiplied by $(-1)$) in Problem 1 of Preview Activity \ref{pa:1_b}.


\item Find the indicated matrix-vector products, if possible. Express as one vector.
    \ba
    \item $\left[ \begin{array}{cr} 2& -2 \\ 1& 2 \end{array} \right] \left[ \begin{array}{r} 1 \\ -1 \end{array} \right]$

    

    \item $\left[ \begin{array}{rrr} 1 & 0 & 2\\ 2 & -2 & 3  \end{array} \right] \left[ \begin{array}{r} 2\\1 \end{array} \right]$

    
		
		\item $\left[ \begin{array}{rrc} -6 & -2 & 1\\ 2 & -2 & 1  \end{array} \right] \left[ \begin{array}{r} 1\\-3 \\1\end{array} \right]$

    

    \ea

\item As you might have noticed, systems with all the constants being 0 are special in that they always have a solution. (Why?) So we might consider grouping systems into two types: Those of the form $A \vx = \vb$, where not all of the entries of the vector $\vb$ are $0$, and those of the form $A \vx = \vzero$, where $\vzero$ is the vector of all zeros. Systems like $A \vx = \vb$, where $\vb$ contains at least one non-zero entry, are called \emph{nonhomogeneous}\index{nonhomogeneous system}  systems, and systems of the form $A \vx = \vzero$ are called \emph{homogeneous}\index{homogeneous system} systems. For every nonhomogeneous system $A \vx = \vb$ there is a corresponding homogeneous system $A \vx = \vzero$, and there is a useful connection between the solutions to the nonhomogeneous system and the corresponding homogeneous system. For example, consider the nonhomogeneous system
\[A \vx = \vb\]
with 
 \begin{equation} \label{eq:PA1e_2} 
A = \left[ \begin{array}{ccc} 1 & 1 & 2  \\ 1 & 2 & 1  \end{array} \right], \ \vx = \left[ \begin{array}{c} x_1 \\ x_2 \\ x_3 \end{array} \right], \ \text{ and } \ \vb =  \left[ \begin{array}{r} 0 \\ -2 \end{array} \right].
\end{equation}
The augmented matrix representation of this system is $[A \ | \ \vb]$. If we reduce this augmented matrix, we find 
\[\left[ \begin{array}{ccr|r} 1 & 0 & 3 & 2 \\ 0 & 1 & -1 & -2 \end{array} \right].\]
From this RREF, we immediately see that the general solution is that $x_3$ is free, $x_2 = x_3-2$, and $x_1 = 2-3x_3$. In vector form, we can represent this general solution as
\begin{equation} \label{eq:PA1e_3}
\left[ \begin{array}{c} x_1 \\ x_2 \\ x_3 \end{array} \right] = \left[ \begin{array}{c} 2-3x_3 \\ x_3-2 \\ x_3 \end{array} \right] = \left[ \begin{array}{r} 2 \\ -2 \\ 0 \end{array} \right] + x_3\left[ \begin{array}{r} -3 \\ 1 \\ 1 \end{array} \right].
\end{equation}
The rightmost expression above is called the \emph{parametric vector form}\index{parametric vector form} of the solution. 

If we had a system where the general solution involved more than one free variable, then we would write the parametric vector form to include one vector multiplying each free variable. For example, if the general solution of a system were that $x_2$ and $x_3$ are free and $x_1=2+x_2+3x_3$, then the parametric vector form would be 
\[ \vx = \left[ \begin{array}{c} 2+ x_2+3x_3 \\ x_2 \\ x_3 \end{array} \right] = \left[ \begin{array}{c} 2 \\ 0 \\ 0 \end{array} \right] + x_2 \left[ \begin{array}{c} 1 \\ 1 \\ 0 \end{array} \right] + x_3 \left[ \begin{array}{c} 3 \\ 0 \\ 1\end{array} \right] \, . \]

Note that the parametric vector form expresses the solutions as a linear combination of a number of vectors, depending on the number of free variables, with an added constant vector. This expression helps us in interpreting the solution set geometrically, as we will see in this section.


	
	\ba
	\item Find the general solution to the homogeneous system 
	\[A \vx = \vzero\]
	with $A$ and $\vx$ as in (\ref{eq:PA1e_2}) and compare it to the solution to the nonhomogeneous system in (\ref{eq:PA1e_3}). What do you notice? 


	\item Find the general solution to the nonhomogeneous system 
	\[A \vx = \vb\]
	with 
\[A = \left[ \begin{array}{ccr} 1&2&-1 \\ 2&4&-2 \end{array} \right], \ \vx =  \left[ \begin{array}{c} x_1 \\ x_2 \\ x_3 \end{array} \right], \ \text{ and } \ \vb =  \left[ \begin{array}{r} -1 \\ 1 \end{array} \right].\]
and express it in parametric vector form. Then find the general solution to the corresponding homogeneous system and express it in parametric vector form. How are the two solution sets related? 


	\item Make a conjecture about the relationship between the solutions to a consistent nonhomogeneous system $A \vx = \vb$ and the corresponding homogeneous system $A \vx = \vzero$. Be as specific as possible.

	
	\ea

\ee

\end{pa} 



\csection{The Matrix-Vector Product}
\label{sec:mv_prod}

The matrix-vector product we defined in Preview Activity \ref{pa:1_e} for a specific example generalizes in a very straightforward manner, and provides a convenient way to represent a system of linear equations of any size using matrices and vectors. In addition to providing us with an algebraic approach to solving systems via matrices and vectors -- leading to a powerful geometric relationship between solution sets of homogeneous and non-homogeneous systems -- this representation allows us to think of a linear system from a dynamic perspective, as we will see later in the section on matrix transformations. 

The matrix-vector product $A \vx$ is a linear combination of the columns of $A$ with weights from $\vx$. To define this product in general, we will need a little notation. Recall that a matrix is made of rows and columns -- the entries reading from left to right form the \emph{rows}\index{matrix!rows} of the matrix and the entries reading from top to bottom form the \emph{columns}\index{matrix!columns}. For example, the matrix  
\[A = \left[ \begin{array}{cccc} 1 & 2 & 3 & 4 \\ 5 & 6 & 7 & 8  \\ 9 & 10 & 11 & 12 \end{array} \right].\]
has three rows and four columns. The number of rows and columns of a matrix is called the \emph{size}\index{matrix!size} of the matrix, so $A$ is a 3 by 4 matrix (also written as $3 \times 4$). We often need to have a way to reference the individual entries of a matrix $A$, and to do so we typically give a label, say $a_{ij}$ to the entry in the $i$th row and $j$th column of $A$. So in our example we have $a_{23}=7$. We also write $A = [a_{ij}]$ to indicate a matrix whose $i,j$th entry is $a_{ij}$. At times it is convenient to write a matrix in terms of its rows or columns. If $A = [a_{ij}]$ is an $m \times n$ matrix, then we will write
\[A = \left[ \begin{array}{ccccc}
a_{11} & a_{12}  & \cdots    & a_{1n-1} & a_{1n} \\
a_{21} & a_{22}  & \cdots    & a_{2n-1} & a_{2n} \\
 \vdots &         & \ddots    &     &\vdots \\
a_{m1} & a_{m2}  & \cdots    & a_{mn-1} & a_{mn}
\end{array} \right]\]
or, if we let $\vr_1, \vr_2, \ldots, \vr_m$ denote the rows of the matrix $A$, then we can write $A$ as\footnote{Technically, the rows of $A$ are made from the entries of the row vectors, but we use this notation as a shorthand.}
\[A = \left[ \begin{array}{c} \vr_1 \\ \vr_2 \\ \vdots \\ \vr_m \end{array} \right].\]
We can also write $A$ in terms of its columns, $\vc_1, \vc_2, \ldots, \vc_n$, as
\[A = [\vc_1 \ \vc_2 \ \cdots \ \vc_n].\]

In general, the product of a matrix with a vector is defined as follows. 



\begin{definition} Let $A$ be an $m \times n$ matrix with columns $\vc_1$, $\vc_2$, $\ldots$, $\vc_n$, and let $\vx = \left[ \begin{array}{c} x_1 \\ x_2 \\ \vdots \\ x_n \end{array} \right]$ be a vector in $\R^n$. The \textbf{matrix-vector product}\index{matrix-vector product} $A\vx$ is
\[A \vx = x_1 \vc_1 + x_2 \vc_2 + \cdots + x_n \vc_n.\]
\end{definition}



\noindent \textbf{Important Note:} The matrix-vector product $A \vx$ is defined only when the number of entries of the vector $\vx$ is equal to the number of columns of the matrix $A$. That is, if $A$ is an $m \times n$ matrix, then $A\vx$ is defined only if $\vx$ is a column vector with $n$ entries. 


\csection{The Matrix-Vector Form of a Linear System}
\label{sec:mv_form}

As we saw in Preview Activity \ref{pa:1_e}, the matrix-vector product provides us with a short hand way of representing a system of linear equations. In general, every linear system can be written in matrix-vector form as follows.

The linear system
\begin{alignat*}{5}
{a_{11}}x_1 	&{}+{} 	&{a_{12}}x_2 	&{}+{}	&\cdots 			&{}+{}	&{a_{1n}}x_n		&{}={}	&b_1 \\
{a_{21}}x_1 	&{}+{} 	&{a_{22}}x_2		&{}+{}	&\cdots			&{}+{}	&{a_{2n}}x_n		&{}={}	&b_2 \\
{} 				&{} 		&{}					&{}		&\vdots \ \		&{}		&{}					&{}		&{} \\
{a_{m1}}x_1 	&{}+{} 	&{a_{2m}}x_2	&{}+{}	&\cdots			&{}+{}	&{a_{mn}}x_n	&{}={}	&b_m 
\end{alignat*} 
of $m$ equations in $n$ unknowns can be written in matrix-vector form as $A\vx = \vb$, where 
\[A = \left[ \begin{array}{cccc}
a_{11} & a_{12} & \cdots & a_{1n} \\
a_{21} & a_{22} & \cdots & a_{2n} \\
\vdots &        & \ddots    & \vdots \\
a_{m1} & a_{m2} & \cdots & a_{mn}
\end{array} \right],  \ \ \ \vx = \left[ \begin{array}{c} x_1 \\ x_2 \\ \vdots \\ x_n \end{array} \right],  \ \ \ \text{ and } \ \ \ \vb = \left[ \begin{array}{c} b_1 \\ b_2 \\ \vdots \\ b_m \end{array} \right].\]
This general system can also be written in the vector form
\[x_1  \left[ \begin{array}{c} a_{11}  \\ a_{21} \\ \vdots \\ a_{m1} \end{array} \right] + x_2 \left[ \begin{array}{c} a_{12}  \\ a_{22} \\ \vdots \\ a_{m2} \end{array} \right]  + \cdots + x_n \left[ \begin{array}{c} a_{1n}  \\ a_{2n}  \\ \vdots \\ a_{mn} \end{array} \right] = \left[ \begin{array}{c} b_1 \\ b_2 \\ \vdots \\ b_m \end{array} \right].\]



With this last representation, we now have four different ways to represent a system of linear equations (as a system of linear equations, as an augmented matrix, in vector equation form, and in matrix-vector equation form), and it is important to be able to translate between them. As an example, the system
\begin{alignat*}{5}
x_1 		&{}+{} 	&{4}x_2 	&{}+{}	&{2}x_3 	&{}+{}	&{4}x_4	&{}={}	&1 \\
{2}x_1 	&{}-{} 	&{}x_2 		&{}-{}	&{5}x_3 	&{}-{}	&{}x_4	&{}={}	&2 \\
{3}x_1 	&{}+{} 	&{7}x_2 	&{}+{}	&{}x_3 		&{}+{}	&{7}x_4	&{}={}	&3 \\
\end{alignat*}
from the introduction to this section has corresponding augmented matrix
\[\left[ \begin{array}{crrr|c} 1&4&2&4&1 \\ 2&-1&-5&-1&2 \\ 3&7&1&7&3 \end{array} \right],\]
is expressed in vector form as 
\[x_1\left[ \begin{array}{c} 1\\2\\3 \end{array} \right] + x_2\left[ \begin{array}{r} 4\\-1\\7 \end{array} \right] + x_3\left[ \begin{array}{r} 2\\-5\\1 \end{array} \right] + x_4\left[ \begin{array}{r} 4\\-1\\7 \end{array} \right]= \left[ \begin{array}{c} 1\\2\\3 \end{array} \right],\]
and has matrix-vector form
\[\left[ \begin{array}{crrr} 1&4&2&4 \\ 2&-1&-5&-1 \\ 3&7&1&7 \end{array} \right] \left[ \begin{array}{c} x_1 \\x_2 \\ x_3 \\ x_4 \end{array} \right] = \left[ \begin{array}{c} 1 \\ 2 \\ 3 \end{array} \right].\]


\begin{activity} In this activity, we will use the equivalence of the different representations of a system to make useful observations about when a system represented as $A\vx=\vb$ has a solution. 

\ba
\item Consider the system
\[\left[ \begin{array}{ccr} 1&2&-1 \\ 2&1&3 \end{array} \right] \left[ \begin{array}{c} x_1 \\ x_2 \\ x_3\end{array} \right] = \left[ \begin{array}{c} 2 \\ 6 \end{array} \right].\]
Write the matrix-vector product on the left side of this equation as a linear combination of the columns of the coefficient matrix. Find weights that make the vector $\left[ \begin{array}{c} 2 \\ 6 \end{array} \right]$ a linear combination of the columns of the coefficient matrix. 

\item From this point on we consider the general case where $A$ is an $m \times n$ matrix. Use the vector equation representation to explain why the system $A \vx = \vb$ has a solution if and only if $\vb$ is a linear combination of the columns of $A$. (Note that `if and only if' is an expression to mean that if one side of the expression is true, then the other side must also be true.) (Hint: Compare to what you did in part (a).)



\item Use part (b) and the definition of span to explain why the system $A \vx = \vb$ has a solution if and only if the vector $\vb$ is in the span of the columns of $A$. 



\item Use part (c) to explain why the system $A \vx = \vb$ always has a solution for any vector $\vb$ in $\R^m$ if and only if the span of the columns of $A$ is all of $\R^m$.



\item Use the augmented matrix representation and the criterion for a consistent system to explain why the system $A \vx = \vb$ is consistent for all vectors $\vb$ if and only if $A$ has a pivot position in every row.



\ea

\end{activity}

We summarize our observations from the above activity in the following theorem.

\begin{theorem} Let $A$ be an $m \times n$ matrix. The following statements are equivalent:
    \begin{enumerate}
    \item The matrix equation $A \vx = \vb$ has a solution for every vector $\vb$ in $\R^m$.
    \item Every vector $\vb$ in $\R^m$ can be written as a linear combination of the columns of $A$.
    \item The span of the columns of $A$ is $\R^m$.
    \item The matrix $A$ has a pivot position in each row.
    \end{enumerate}
\end{theorem}



In the future, if we need to determine whether a system has a solution for every $\vb$, we can refer to this theorem without having to argue our reasoning from scratch.



\subsection*{Properties of the Matrix-Vector Product}

As we have done before, we have a new operation (the matrix-vector product), so we should wonder what properties it has. 


\begin{activity} \label{act:A1.4_8} In this activity, we consider whether the matrix-vector product distributes vector addition. In other words: Is $A(\vu + \vv)$ equal to $A\vu + A\vv$?\\

We work with arbitrary vectors $\vu, \vv$ in $\R^3$ and an arbitrary matrix $A$ with 3 columns (so that $A\vu$ and $A\vv$ are defined) to simplify notation. Let $A = [\vc_1 \ \vc_2 \ \vc_3]$ (note that each $\vc_i$ represents a column of $A$), $\vu = \left[ \begin{array}{c} u_1 \\ u_2 \\ u_3 \end{array} \right]$, and $\vv = \left[ \begin{array}{c} v_1 \\ v_2 \\ v_3 \end{array} \right]$. Use the definition of the matrix-vector product along with the properties of vector operations to show that
\[A(\vu + \vv)= A\vu + A\vv.\]


\end{activity}



Similar arguments using the definition of matrix-vector product along with the properties of vector operations can be used to show the following theorem:


\begin{theorem} \label{thm:IMT_1_e} Let $A$ be an $m \times n$ matrix, $\vu$ and $\vv$ $n \times 1$ vectors, and $c$ a scalar. Then
\begin{enumerate}
\item $A(\vu + \vv) = A\vu + A\vv$
\item $c(A\vv) = A(c\vv)$
\end{enumerate}
\end{theorem}

\csection{Homogeneous and Nonhomogeneous Systems}
\label{sec:homog_sys}

As we saw before, the systems with all the right hand side constants being 0 are special in that they always have a solution. (Why?) So we might consider grouping systems into two types: Those of the form $A \vx = \vb$, where not all of the entries of the vector $\vb$ are $0$, and those of the form $A \vx = \vzero$, where $\vzero$ is the vector of all zeros. Systems like $A \vx = \vb$, where $\vb$ contains at least one non-zero entry, are called \textbf{nonhomogeneous} systems, and systems of the form $A \vx = \vzero$ are called \textbf{homogeneous} systems. For every nonhomogeneous system $A \vx = \vb$ there is a corresponding homogeneous system $A \vx = \vzero$. We now investigate the connection between the solutions to the nonhomogeneous system and the corresponding homogeneous system. 

\vspace{0.2cm}

\begin{activity} In this activity we will consider the relationship between the solution sets of nonhomogeneous systems and those of the corresponding homogeneous systems.

\ba 
\item Find the solution sets of the system
\[A \vx = \vb\]
where 
\[ A = \left[ \begin{array}{ccc} 1 & 1 & 2  \\ 1 & 2 & 1  \end{array} \right], \ \vx = \left[ \begin{array}{c} x_1 \\ x_2 \\ x_3 \end{array} \right], \ \text{ and } \ \vb =  \left[ \begin{array}{r} 0 \\ -2 \end{array} \right] \]
and the corresponding homogeneous system (i.e. where we replace $\vb$ with $\vzero$.)


\item Find the solution sets of the system
\[A \vx = \vb\]
where 
\[A = \left[ \begin{array}{ccr} 1&2&-1 \\ 2&4&-2 \end{array} \right], \ \vx =  \left[ \begin{array}{c} x_1 \\ x_2 \\ x_3 \end{array} \right], \ \text{ and } \ \vb =  \left[ \begin{array}{r} -1 \\ 1 \end{array} \right]\]
and the corresponding homogeneous system.


\item What are the similarities/differences between solutions of the nonhomogeneous system and its homogeneous counterpart?


\ea
\end{activity}


As we saw in the above activity, there is a relationship between solutions of a nonhomogeneous and the corresponding homogeneous system. Let us formalize this relationship. If the general solution of a system involves free variables, we can represent the solutions in \textbf{parametric vector form} to have a better idea about the geometric representation of the solution set. Suppose the solution is that $x_3$ is free, $x_2 = -2+x_3$, and $x_1 = 2-3x_3$. In vector form, we can represent this general solution as
\begin{equation} 
\left[ \begin{array}{c} x_1 \\ x_2 \\ x_3 \end{array} \right] = \left[ \begin{array}{c} 2-3x_3 \\ x_3-2 \\ x_3 \end{array} \right] = \left[ \begin{array}{r} 2 \\ -2 \\ 0 \end{array} \right] + x_3\left[ \begin{array}{r} -3 \\ 1 \\ 1 \end{array} \right].
\end{equation}
From this representation, we see that the solution set is a line through the origin (formed by multiples of $\left[ \begin{array}{r} -3 \\ 1 \\ 1 \end{array} \right]$) shifted by the added vector $\left[ \begin{array}{r} 2 \\ -2 \\ 0 \end{array} \right]$. The solution to the homogeneous system on the other does not have the shift.

\vspace{0.2cm}

Algebraically, we see that every solution to the nonhomogeneous system $A\vx = \vb$ can be written in the form $\vp + \vv_h$, where $\vp$ is a particular solution to $A\vx = \vb$ and $\vv_h$ is a solution to the corresponding homogeneous system $A\vx = \vzero$. 

\vspace{0.2cm}

To understand why this \underline{always} happens, we will verify the result algebraically for an arbitrary $A$ and $\vb$. Assuming that $\vp$ is a particular solution to the nonhomogeneous system $A\vx=\vb$, we need to show that:
\begin{itemize}
\item if $\vv$ is an arbitrary solution to the nonhomogeneous system, then $\vv = \vp + \vv_h$, where $\vv_h$ is some solution to the homogeneous system $A\vx = \vzero$, and
\item if $\vv_h$ is an arbitrary solution to the homogeneous system, then $\vp + \vv_h$ is a solution to the nonhomogeneous system.
\end{itemize}

To verify the first condition, suppose that $\vv$ is a solution to the nonhomogeneous system $A \vx = \vb$. Since we want $\vv=\vp+\vv_h$, we need to verify that $\vv-\vp$ is a solution for the homogeneous system so that we can assign $\vv_h=\vv-\vp$. Note that
\[A(\vv - \vp) = A\vv - A\vp = \vb - \vb = \vzero \, ,\]
using the distributive property of matrix-vector product over vector addition. Hence $\vv$ is of the form $\vp + \vv_h$ with $\vv_h = \vzero$.

To verify the second condition, consider a vector of the form $\vp + \vv_h$, where $\vv_h$ is a homogeneous solution. We have
\[A(\vp + \vv_h) = A\vp + A\vv_h = \vb + \vzero = \vb,\]
and so $\vp + \vv_h$ is a solution to $A \vx = \vb$.



Our work above proves the following theorem.

\begin{theorem} \label{thm:1_e_1} Suppose the equation $A\vx=\vb$ is consistent for some $\vb$ and $\vp$ is a solution. Then the solution set of $A\vx=\vb$ consists of all vectors of the form $\vv=\vp+\vv_h$ where $\vv_h$ is a solution to $A\vx=\vzero$.
\end{theorem}


\csection{The Geometry of Solutions to the Homogeneous System}
\label{sec:geom_homog_sys}

There is a simple geometric interpretation to the solution set of the homogeneous system $A \vx = \vzero$ based on the number of free variables that imposes a geometry on the solution set of the corresponding nonhomogeneous system $A \vx = \vb$ (when consistent) due to Theorem \ref{thm:1_e_1}.



\begin{activity} \label{ex:Homogeneous2} In this activity we consider geometric interpretations of the solution sets of homogeneous and nonhomogeneous systems.

\ba \item Consider the system $A \vx = \vb$ where $A = \left[ \begin{array}{rr} 1 & -3 \\ -3 & 9 \\ -1 & 3 \end{array} \right]$ and $\vb =  \left[ \begin{array}{r} 2 \\ -6 \\ -2 \end{array} \right]$. The general solution to this system has the form $\left[ \begin{array}{c} 2 \\ 0 \end{array} \right] + x_2\left[ \begin{array}{c} 3 \\ 1 \end{array} \right]$, where $x_2$ is any real number.
\begin{enumerate}[i.]
    \item Let $\vv = \left[ \begin{array}{c} 3 \\ 1 \end{array} \right]$. What does the set of all vectors of the form $x_2\vv$ look like geometrically? Draw a picture in $\R^2$ to illustrate. (Recall that we refer to all the vectors of the form $x_2\vv$ simply as $\Span \{\vv\}$.)

    

    \item Let $\vp = \left[ \begin{array}{r} 2 \\ 0 \end{array} \right]$. What effect does adding the vector $\vp$ to each vector in $\Span \{\vv\} $ have on the geometry of $\Span \{\vv\}$? Finally, what does this mean about the geometry of the solution set to the nonhomogeneous system $A \vx = \vb$?

    
		
\end{enumerate}

\item Consider the system $A \vx = \vb$ where $A =\left[ \begin{array}{ccr} 1&2&-1 \\ 3&6&-3 \end{array} \right]$ and $\vb =  \left[ \begin{array}{r} -2 \\ -6 \end{array} \right]$. The general solution to this system has the form $\left[ \begin{array}{r} -2 \\ 0 \\0 \end{array} \right] + x_2\left[ \begin{array}{r} -2 \\ 1 \\0 \end{array} \right] +x_3\left[ \begin{array}{c} 1 \\ 0 \\1 \end{array} \right] $, where $x_2, x_3$ are any real numbers. 

\begin{enumerate}[i.]
    \item Let $\vu = \left[ \begin{array}{r} -2 \\ 1 \\0 \end{array} \right], \vv= \left[ \begin{array}{c} 1 \\ 0 \\1 \end{array} \right]$. Use our results from Section 4 to determine the geometric shape of $\Span \{\vu, \vv\}$, the set of all vectors of the form $x_2\left[ \begin{array}{r} -2 \\ 1 \\0 \end{array} \right] +x_3\left[ \begin{array}{c} 1 \\ 0 \\1 \end{array} \right]$, where $x_2, x_3$ are any real numbers. 

    

    \item Let $\vp = \left[ \begin{array}{r} -2 \\ 0 \\0 \end{array} \right]$. What's the geometric effect of adding the vector $\vp$ to each vector in $\Span \{\vu, \vv\} $? Finally, what does this mean about the geometry of the solution set to the nonhomogeneous system $A \vx = \vb$?

\end{enumerate}

   \ea
\end{activity}



Our work in the above activity shows the geometric shape of the solution set of a consistent nonhomogeneous system is the same as the geometric shape of the solution set of the corresponding homogeneous system. The only difference between the two solution sets is that one is a shifted version of the other.






\csection{Examples}
\label{sec:mv_form_exam}

\ExampleIntro

\begin{example} We now have several different ways to represent a system of linear equations.  Rewrite the system in an equivalent form 
\begin{alignat*}{6}
{11}x_1	&{}+{} 	&{4}x_2 	&{}-{}	&{5}x_3 	&{}-{}	&{2}x_4	&= &{} &63&{} \\
{15}x_1 	&{}+{} 	&{5}x_2 	&{}+{}	&{2}x_3 	&{}-{}	&{2}x_4	&= &{} &68&{} \\
{6}x_1 	&{}+{} 	&{2}x_2 	&{}+{}	&{}x_3 	&{}-{}	&{}x_4	&= &{} &26&{} \\
{9}x_1 	&{}+{} 	&{3}x_2 	&{}+{}	&{2}x_3 	&{}-{}	&{}x_4	&= &{} &40&{.} 
\end{alignat*}
	\ba
	\item as an augmented matrix
	\item as an equation involving a linear combination of vectors
	\item using a matrix-vector product
	\ea
Then solve the system. \\

\ExampleSolution
\ba
\item The augmented matrix for this system is 
\[\left[ \begin{array}{ccrr|c} 11&4&-5&-2&63 \\ 15&5&2&-2&68 \\ 6&2&1&-1&26 \\ 9&3&2&-1&40 \end{array} \right].\]

\item If we make vectors from the columns of the augmented matrix, we can write this system in vector form as
\[x_1 \left[ \begin{array}{c} 11 \\ 15 \\ 6 \\ 9 \end{array} \right] + x_2 \left[ \begin{array}{c} 4 \\ 5 \\ 2 \\ 3 \end{array} \right] + x_3 \left[ \begin{array}{r} -5 \\ 2\\ 1 \\ 2 \end{array} \right] + x_4 \left[ \begin{array}{r} -2\\ -2 \\ -1 \\ -1 \end{array} \right] = \left[ \begin{array}{c} 63 \\ 68 \\ 26 \\ 40 \end{array} \right].\]

\item The coefficient matrix for this system is $\left[ \begin{array}{ccrr} 11&4&-5&-2 \\ 15&5&2&-2 \\ 6&2&1&-1 \\ 9&3&2&-1 \end{array} \right]$, and the matrix-vector form of the system is 
\[\left[ \begin{array}{ccrr} 11&4&-5&-2 \\ 15&5&2&-2 \\ 6&2&1&-1 \\ 9&3&2&-1 \end{array} \right] \left[ \begin{array}{c} x_1\\x_2\\x_3\\x_4 \end{array} \right] = \left[ \begin{array}{c} 63 \\ 68 \\ 26 \\ 40 \end{array} \right].\]
\ea

Using technology, we find that the reduced row echelon form of the augmented matrix for this system is 
\[\left[ \begin{array}{cccc|r} 1&0&0&0&3 \\ 0&1&0&0&7 \\ 0&0&1&0&-2  \\ 0&0&0&1&4 \end{array} \right].\]
So the solution to this system is $x_1 = 3$, $x_2 = 7$, $x_3 = -2$, and $x_4 = 4$. 


\end{example}

\begin{example} Consider the homogeneous system
\begin{alignat*}{5}
{}x_1		&{}+{} 	&{8}x_2	 &{}-{}	&{}x_3 		&= &{} &0&{} \\
{}x_1 	&{}-{} 	&{7}x_2 	&{}+{}	&{2}x_3 		&= &{} &0&{} \\
{3}x_1 	&{}+{} 	&{4}x_2 	&{}+{}	&{}x_3 		&= &{} &0&{.} 
\end{alignat*}
\ba
\item Find the general solution to this homogeneous system and express the system in parametric vector form. 

\item Let $A = \left[ \begin{array}{crr} 1&8&-1 \\ 1&-7&2 \\ 3&4&1 \end{array} \right]$, and let $\vb = \left[ \begin{array}{r} -6\\9\\2 \end{array} \right]$. Show that $\left[ \begin{array}{r} -1\\0\\5 \end{array} \right]$ is a solution to the non-homogeneous system $A \vx = \vb$. 

\item Use the results from part (a) and (b) to write the parametric vector form of the general solution to the non-homogeneous system $A \vx = \vb$. (Do this without directly solving the system $A \vx = \vb$.) 

\item Describe what the general solution to the homogeneous system $A \vx = \vzero$ and the general solution to the non-homogeneous system $A \vx = \vb$ look like geometrically. 

\ea 

\ExampleSolution
\ba
\item The augmented matrix of the homogeneous system is 
\[ \left[ \begin{array}{crr|c} 1&8&-1&0 \\ 1&-7&2&0 \\ 3&4&1&0 \end{array} \right],\]
and the reduced row echelon form of this augmented matrix is 
\[ \left[ \renewcommand{\arraystretch}{1.3} \begin{array}{ccr|c} 1&0&\frac{3}{5}&0 \\ &1&-\frac{1}{5}&0 \\ 0&0&0&0 \end{array} \right].\]
Since there is no corresponding equation of the form $0 = b$ for a nonzero constant $b$, this system is consistent. The third column contains no pivot, so the variable $x_3$ is free, $x_2 = \frac{1}{5}x_3$ and $x_1 =  -\frac{3}{5}x_3$.
In parametric vector form the general solution to the homogeneous system is 
\[\left[\begin{array}{c} x_1\\x_2\\x_3 \end{array} \right] = \left[\renewcommand{\arraystretch}{1.4} \begin{array}{c} -\frac{3}{5}x_3\\ \frac{1}{5}x_3\\x_3 \end{array} \right]  = x_3 \left[\renewcommand{\arraystretch}{1.4} \begin{array}{r} -\frac{3}{5}\\ \frac{1}{5}\\1 \end{array} \right].\]
\item  Since
\begin{align*}
A \left[ \begin{array}{r} -1\\0\\5 \end{array} \right] &= (-1)\left[ \begin{array}{c} 1\\1\\3 \end{array} \right] + (0)\left[ \begin{array}{r} 8\\-7\\4 \end{array} \right] + (5) \left[ \begin{array}{r} -1\\2\\1 \end{array} \right] \\
	&= \left[ \begin{array}{c} -1-5\\-1+10\\-3+5 \end{array} \right] = \left[ \begin{array}{r} -6\\9\\2 \end{array} \right],
\end{align*}
we conclude that $\left[ \begin{array}{r} -1\\0\\5 \end{array} \right]$ is a solution to the non-homogeneous system $A \vx = \vb$.

\item We know that every solution to the non-homogeneous system $A \vx = \vb$ has the form of the general solution to the homogeneous system plus a particular solution to the non-homogeneous system. Combining the results of (a) and (b) we see that the general solution to the non-homogeneous system $A \vx = \vb$ is
\[\left[\begin{array}{c} x_1\\x_2\\x_3 \end{array} \right] = \left[\begin{array}{r} -1\\0\\5 \end{array} \right]  +  x_3 \left[\renewcommand{\arraystretch}{1.4} \begin{array}{r} -\frac{3}{5}\\ \frac{1}{5}\\1 \end{array} \right],\]
where $x_3$ can be any real number. 

\item The solution to the homogeneous system $A \vx = \vzero$ is the span of the vector $\left[\renewcommand{\arraystretch}{1.4} \begin{array}{r} -\frac{3}{5}\\ \frac{1}{5}\\1 \end{array} \right]$. Geometrically, this set of points is a line through the origin and the point $(-3, 1, 5)$ in $\R^3$. The solution to the non-homogeneous system $A \vx = \vb$ is the translation of the line through the origin and $(-3, 1, 5)$ by the vector $\left[ \begin{array}{r} -1\\0\\5 \end{array} \right]$. In other words, the solution to the non-homogeneous system $A \vx = \vb$ is the line in $\R^3 $ through the points $(-1,0,5)$ and $(-4,1,10)$. 

\ea

\end{example}

\csection{Summary}
\label{sec:mv_form_summ}

\begin{itemize}
\item If $A = [\vc_1 \ \vc_2 \ \cdots \ \vc_n]$ is an $m \times n$ matrix with columns $\vc_1$, $\vc_2$, $\ldots$, $\vc_n$, and if $\vx = \left[ \begin{array}{c} x_1\\x_2\\ \vdots \\ x_n \end{array} \right]$ is a vector in $\R^n$, then the matrix-vector product $A \vx$ is defined to be the linear combination of the columns of $A$ with corresponding weights from $\vx$ -- that is 
\[A \vx = x_1 \vc_1 + x_2 \vc_2 + \cdots + x_n \vc_n.\]
\item A linear system

\begin{alignat*}{5}
{a_{11}}x_1 	&{}+{} 	&{a_{12}}x_2 	&{}+{}	&\cdots 			&{}+{}	&{a_{1n}}x_n		&{}={}	&b_1 \\
{a_{21}}x_1 	&{}+{} 	&{a_{22}}x_2		&{}+{}	&\cdots			&{}+{}	&{a_{2n}}x_n		&{}={}	&b_2 \\
{} 				&{} 		&{}					&{}		&\vdots \ \		&{}		&{}					&{}		&{} \\
{a_{m1}}x_1 	&{}+{} 	&{a_{2m}}x_2	&{}+{}	&\cdots			&{}+{}	&{a_{mn}}x_n	&{}={}	&b_m 
\end{alignat*} 
can be written in matrix form as
\[A \vx = \vb,\]
where
\[A = \left[ \begin{array}{cccc}
a_{11} & a_{12} & \cdots & a_{1n} \\
a_{21} & a_{22} & \cdots & a_{2n} \\
\vdots &        & \ddots    & \vdots \\
a_{m1} & a_{m2} & \cdots & a_{mn}
\end{array} \right],  \ \ \ \vx = \left[ \begin{array}{c} x_1 \\ x_2 \\ \vdots \\ x_n \end{array} \right],  \ \ \ \text{ and } \ \ \ \vb = \left[ \begin{array}{c} b_1 \\ b_2 \\ \vdots \\ b_m \end{array} \right].\]
\item The matrix equation $A \vx = \vb$ has a solution if and only if $\vb$ is a linear combination of the columns of $A$.
\item The system $A \vx = \vb$ is consistent for every vector $\vb$ if every row of $A$ contains a pivot. 
\item A homogeneous system is a system of the form $A \vx = \vzero$ for some $m \times n$ matrix $A$. Since the zero vector in $\R^n$ satisfies $A \vx = \vzero$, a homogeneous system is always consistent.
\item A homogeneous system can have one or infinitely many different solutions. The homogeneous system $A \vx = \vzero$ has exactly one solution if and only if each column of $A$ is a pivot column.
\item The solutions to the consistent nonhomogeneous system $A \vx = \vb$ have the form $\vp + \vv_h$, where $\vp$ is a particular solution to the nonhomogeneous system $A \vx = \vb$ and $\vv_h$ is a solution to the homogeneous system $A \vx = \vzero$. In other words, the solution space to a consistent nonhomogeneous system $A \vx = \vb$ is a translation of the solution space of the homogeneous system $A \vx = \vzero$ by a particular solution to the nonhomogeneous system.
\end{itemize}

Finally, we argued an important theorem.

\noindent \textbf{Theorem 5.2.} Let $A$ be an $m \times n$ matrix. The following statements are equivalent.
    \begin{enumerate}
    \item The matrix equation $A \vx = \vb$ has a solution for every vector $\vb$ in $\R^m$.
    \item Every vector $\vb$ in $\R^m$ can be written as a linear combination of the columns of $A$.
    \item The span of the columns of $A$ is $\R^m$.
    \item The matrix $A$ has a pivot position in each row.
    \end{enumerate}


We will continue to add to this theorem, so it is a good idea for you to begin now to remember the equivalent conditions of this theorem.




\csection{Exercises}
\label{sec:mv_form_exer}

\be
\item Write the system 
\begin{alignat*}{5}
x_1 		&{}+{} 	&{2}x_2 	&{}+{}	&{2}x_3 	&{}+{}	&{}x_4		&{}={}	&{}- &1 \\
{4}x_1 	&{}-{} 	&{8}x_2		&{}+{}	&{3}x_3 	&{}-{}	&{9}x_4		&{}={}	&{} &2 \\
{}x_1 	&{}+{} 	&{6}x_2 	&{}-{}	&{4}x_3		&{}+{}	&{12}x_4	&{}={}	&{}- &1 
\end{alignat*}
in matrix-vector form. Explicitly identify the coefficient matrix and the vector of constants. 

\item Write the linear combination 
\[x_1\left[ \begin{array}{c} 1\\5 \end{array} \right] + x_2\left[ \begin{array}{r} -3\\10 \end{array} \right] + x_3\left[ \begin{array}{c} 2\\2 \end{array} \right] \]
as a matrix-vector product.


\item Represent the following matrix-vector equation as a linear system and find its solution.
\[ \left[ \begin{array}{crc} 2 & 3 & 4 \\ 1 & -2 & 3  \end{array} \right] \left[ \begin{array}{c} x_1 \\ x_2 \\ x_3 \end{array} \right] = \left[ \begin{array}{r} 4\\-6 \end{array} \right] \]

\item Represent the following matrix-vector equation as a linear system and find its solution.
\[ \left[ \begin{array}{crr} 1 & -2 & -1 \\ 2 & 2 & -2 \\ 3 & 1 & 1  \end{array} \right] \left[ \begin{array}{c} x_1 \\ x_2 \\ x_3 \end{array} \right] = \left[ \begin{array}{r} 1\\ -4\\8 \end{array} \right] \]

\item \label{ex:1_e_scalar_product} Another way of defining the matrix-vector product uses the concept of the \emph{scalar product} of vectors.\begin{footnote}{Note that some authors refer to the scalar product as the \emph{dot product}.}\end{footnote} Given a $1 \times n$ matrix $\vu = [u_1 \ u_2 \ \ldots \ u_n]$\footnote{We can identify a $1 \times n$ matrix $\vu = [u_1 \ u_2 \ \ldots \ u_n]$ with the $n \times 1$ vector $\vu=\left[ \begin{array}{c} u_1\\ u_2\\ \vdots \\ u_n \end{array} \right]$, so we ofter refer to $ [u_1 \ u_2 \ \ldots \ u_n]$ as a vector.} and an $n \times 1$ vector $\vv=\left[ \begin{array}{c} v_1\\ v_2\\ \vdots \\ v_n \end{array} \right]$, we define the scalar product $\vu \cdot \vv$ as 
\[\vu \cdot \vv = u_1v_1 + u_2v_2 + u_3v_3 + \cdots + u_nv_n.\]
We then define the matrix-vector product $A\vx$ as the vector whose entries are the scalar products of the rows of $A$ with $\vx$. As an example, if $A= \left[ \begin{array}{crc} 2 & 3 & 4 \\ 1 & -2 & 3 \end{array} \right]$ and $\vx = \left[ \begin{array}{c} x_1 \\ x_2 \\ x_3 \end{array} \right]$, then 
\[A \vx = \left[ \begin{array}{c} 2x_1+3x_2 +4x_3 \\ x_1 +(-2)x_2 + 3x_3 \end{array} \right].\]
Calculate the matrix-vector product $A\vx$ where $A=\left[ \begin{array}{cc} a & b\\ c& d \end{array} \right]$ and $\vx=\left[ \begin{array}{c} x_1 \\ x_2 \end{array} \right]$ using both methods of finding the matrix-vector product to show that the two definitions are equivalent for size $2\times 2$ matrices.

\item Find the value of $a$ such that 
\[ \left[ \begin{array}{crc} 1 & 2 & 2 \\1 & -1 & 3 \\1 & 2 & 4
\end{array} \right] \left[ \begin{array}{r} 1 \\ -1 \\ a \end{array} \right] = \left[ \begin{array}{r} * \\ -5 \\ * \end{array} \right]\]
where $*$'s represent unknown values.

\item Suppose we have 
\[ \left[ \begin{array}{rccc} 1 & 2 & 1 & 2\\-1 & 2 & 3 & 1\\2&3 & 1 & a
\end{array} \right] \left[ \begin{array}{r} 1 \\ 2 \\ -2 \\ 3 \end{array} \right] = \left[ \begin{array}{c} b_1 \\ b_2\\ b_3 \end{array} \right] \]
where $b_i$'s represent unknown values.
	\ba
	\item In order to find the value of $a$, which of the $b_i$'s do we need to know? Why?
	\item Suppose the $b_i$(s) that we need to know is(are) equal to 9. What is the value of $a$?
	\ea

\item Suppose we are given 
\[ A \vu = \left[ \begin{array}{c} 1 \\ 1 \end{array} \right] \; \text{ and } \; A \vv = \left[ \begin{array}{c} 1 \\ 3 \end{array} \right] \]
for an unknown $A$ and two unknown vectors $\vu, \vv$ in $\R^3$. Using matrix-vector product properties, evaluate $A\vw$ where $\vw=2\vu-3\vv$.

\item Suppose we are given 
\[ A \left[ \begin{array}{c} 1 \\ 2\\ 1 \end{array} \right] = \left[ \begin{array}{c} 1 \\ 1 \end{array} \right] \; \text{ and } \; A \left[ \begin{array}{c} 1 \\ 0 \\2  \end{array} \right] = \left[ \begin{array}{c} 0 \\ 2 \end{array} \right] \, .\]
After expressing $\left[ \begin{array}{r} -1 \\ 6 \\ -5 \end{array} \right]$ as a linear combination of $\left[ \begin{array}{c} 1 \\ 2\\ 1 \end{array} \right]$ and $\left[ \begin{array}{c} 1 \\ 0 \\2  \end{array} \right]$, use the matrix-vector product properties to determine $A\left[ \begin{array}{r} -1 \\ 6 \\ -5 \end{array} \right]$. 

\item 
	\ba
	\item The non-homogeneous system (with unknown constants $a$ and $b$)
\begin{alignat*}{4}
{}x 	&{}+{} 	&{}y 	&{}-{}	&{}z		&{}={}	&2 \\
{2}x 	&{}+{} 	&{a}y 	&{}+{}	&{b}z	&{}={}	&4
\end{alignat*} 
has a solution which lies on the $x$-axis (i.e. $y=z=0$). Find this solution.

	\item If the corresponding homogeneous system
\begin{alignat*}{4}
{}x 	&{}+{} 	&{}y 	&{}-{}	&{}z		&{}={}	&0 \\
{2}x 	&{}+{} 	&{a}y 	&{}+{}	&{b}z	&{}={}	&0
\end{alignat*} 
has its general solution expressed in parametric vector form as $z\cdot \left[ \begin{array}{c} 0\\1\\1 \end{array} \right]$, find the general solution for the non-homogeneous system using your answer to part (a).

	\item Find the conditions on $a$ and $b$ that make the system from (a) have the general solution you found in (b). 
	

	\ea
	
\item Find the general solution to the non-homogeneous system
\begin{alignat*}{5}
{}x 		&{}-{} 	&{2}y 	&{}+{}	&{}z		&{}={}	&{}&3&{} \\
{-2}x	&{}+{} 	&{4}y 	&{}-{}	&{2}z	&{}={}	&{-}&6&{.}
\end{alignat*} 
Using the parametric vector form of the solutions, determine what the solution set to this non-homogeneous system looks like geometrically. Be as specific as possible. (Include information such as whether the solution set is a point, a line, or a plane, etc.; whether the solution set passes through the origin or is shifted from the origin in a specific direction by a specific number of units; and how the solution is related to the corresponding homogeneous system.)

\item Come up with an example of a $3\times 3$ matrix $A$ for which the solution set of $A\vx=\vzero$ is a line, and a $3\times 3$ matrix $A$ for which the solution set of $A\vx=\vzero$ is a plane.

\item Suppose we have three vectors $\vv_1, \vv_2$ and $\vv_3$ satisfying $\vv_3=2\vv_1-\vv_2$. Let $A$ be the matrix with vectors $\vv_1, \vv_2$ and $\vv_3$ as the columns in that order. Find a non-zero $\vx$ such that $A\vx=\vzero$ using this information.

\item Label each of the following statements as True or False. Provide justification for your response.
\ba
\item \textbf{True/False} If the system $A\vx = \vzero$ has infinitely many solutions, then so does the system $A\vx=\vb$ for \textbf{any} right-hand-side $\vb$.

\item \textbf{True/False} If $\vx_1$ is a solution for $A\vx=\vb_1$ and $\vx_2$ is a solution for $A\vx=\vb_2$, then $\vx_1+\vx_2$ is a solution for $A\vx=\vb_1+\vb_2$.

\item \textbf{True/False} If an $m \times n$ matrix $A$ has a pivot in every row, then the equation $A\vx=\vb$ has a unique solution for every $\vb$.

\item \textbf{True/False} If an $m \times n$ matrix $A$ has a pivot in every row, then the equation $A\vx=\vb$ has a solution for every $\vb$.

\item \textbf{True/False} If $A$ and $B$ are row equivalent matrices and the columns of $A$ span $\R^m$, then so do the columns of $B$.

\item \textbf{True/False} All homogeneous systems have either a unique solution or infinitely many solutions.

\item \textbf{True/False} If a linear system is not homogeneous, then the solution set does not include the origin.

\item \textbf{True/False} If a solution set of a linear system does not include the origin, the system is not homogeneous.

\item \textbf{True/False} If the system $A\vx=\vb$ has a unique solution for some $\vb$, then the homogeneous system has only the trivial solution.

\item \textbf{True/False} If $A$ is a $3\times 4$ matrix, then the homogeneous equation $A\vx=\vzero$ has non-trivial solutions.

\item \textbf{True/False}  If $A$ is a $3\times 2$ matrix, then the homogeneous equation $A\vx=\vzero$ has non-trivial solutions.

\ea
\ee

\csection{Project: Input-Output Models}
\label{sec:proj_io_models}

There are two basic types of input-output models\index{input-output models}: closed and open. The closed model assumes that all goods produced are consumed within the economy -- no trading takes place with outside entities. In the open model, goods produced within the economy can be traded outside the economy.

To work with a closed model, we use an example (from \emph{Input-Output Economics} by Wassily Leontief). Assume a simple three-sector economy consisting of agriculture (growing wheat), manufacturing (producing cloth), and households (supplying labor). Each sector of the economy relies on goods from the other sectors to operate (e.g., people must eat to work and need to be clothed). To model the interactions between the sectors, we consider how many units of product is needed as input from one sector to another to produce one unit of product in the second sector. For example, assume the following:
\begin{itemize}
\item to produce one unit (say dollars worth) of agricultural goods requires 25\% of a unit of agricultural output, 28\% of a unit of manufacturing output, and 27\% of a unit of household output;
\item to produce one unit of manufactured goods requires 20\% of a unit of agricultural output, 60\% of a unit of manufacturing output, and 60\% of a unit of household output;
\item to produce one unit of household goods requires 55\% of a unit of agricultural output, 12\% of a unit of manufacturing output, and 13\% of a unit of household output.
\end{itemize}
These assumptions are summarized in Table~\ref{T:ThreeSectorTable}.

\begin{table}[h]
\begin{center}
\begin{tabular}{l | c c c } \\
into\verb+\+from    &Agriculture    & Manufacture   &Households \\ \hline
Agriculture         &0.25           &0.28           &0.27 \\
Manufacture         &0.20           &0.60           &0.60 \\
Households          &0.55           &0.12           &0.13 \\
\end{tabular}
\end{center}
\caption{Summary of simple three sector economy.}
\label{T:ThreeSectorTable}
\end{table}
%\verb*http://books.google.com/books?id=hBDEXblq6HsC&dq=Leontief+Input-output&printsec=frontcover&source=bl&ots=9hTYFNbQ6R&sig=h2F3gSNDNgdYUsTZXXpvYVuXXQw&hl=en&ei=Ztd1Su2DBpCCMfSWyLEM&sa=X&oi=book_result&ct=result&resnum=9#v=onepage&q=&f=false*
%From Table 2-1
This model is said to be \emph{closed} because all good produced are used up within the economy.  If there are goods that are not used within the economy the model is said to be \emph{open}. Open models will be examined later.


The economist's goal is to determine what level of production in each section meets the following requirements:
\begin{itemize} \setlength{\itemsep}{-3pt}
\item the production from each sector meets the needs of all of the sectors and
\item there is no overproduction.
\end{itemize}

\begin{pactivity} \label{act:Closed_model} We can use techniques from linear algebra to determine the levels of production that precisely meet the two goals of the economist.
	\ba
	\item Suppose that the agricultural output is $x_1$ units, the manufacturing output is $x_2$ units, and the household output is $x_3$ units.  We represent this data as a \emph{production vector} $\left[ \begin{array}{c} x_1 \\ x_2 \\ x_3 \end{array} \right]$. To produce a unit of agriculture requires 0.25 units of agriculture, $0.28$ units of manufacturing, and $0.27$ units of household. If $x_1$ units of agriculture, $x_2$ units of manufacturing, and $x_3$ units of household products are are produced, then agriculture can produce  
\[0.25x_1 + 0.28x_2 + 0.27x_3\]
units. In order to meet the needs of agriculture and for there to be no overproduction, we must then have 
\[0.25x_1 + 0.28x_2 + 0.27x_3 = x_1.\] 
 Write similar equations for the manufacturing and household sectors of the economy.


\item  Find the augmented matrix for the system of linear equations that represent production of the three sectors from part (a), and then solve the system to find the production levels that meet the economist's two goals. 


\item Suppose the production level of the household sector is 200 million units (dollars). Find the production levels of the agricultural and manufacturing sectors that meet the economist's two goals.


\ea
\end{pactivity}

In general, a matrix derived from a table like Table~\ref{T:ThreeSectorTable} is called a \emph{consumption} matrix, which we will denote as $C$. (In the example discussed here $C = \left[\begin{array}{ccc} 0.25 & 0.28 & 0.27 \\ 0.20 &0.60 &0.60 \\ 0.55 &0.12 &0.13 \end{array}\right]$.)  A consumption matrix $C = [c_{ij}]$, where $c_{ij}$ represents the proportion of the output of sector $j$ that is consumed by sector $i$, satisfies two important properties.
\begin{itemize}
\item Since no sector can consume a negative amount or an amount that exceeds the output of another sector, we must have $0 \leq c_{ij} \leq 1$ for all $i$ and $j$.
\item If there are $n$ sectors in the economy, the fact that all output is consumed within the economy implies that $c_{1j}+c_{2j}+ \cdots + c_{nj} = 1$. In other words, the column sums of $C$ are all 1. 
\end{itemize}
In our example, if we let $x = \left[\begin{array}{c} x_1 \\ x_2 \\ x_3 \end{array}\right]$, then we can write the equations that guarantee that the production levels satisfy the two economists' goal in matrix form as
\begin{equation} \label{eq:closed}
\vx = C\vx.
\end{equation}
Now we can rephrase the question to be answered as which production vectors $\vx$ satisfy equation (\ref{eq:closed}). When $C\vx = \vx$, then the system is in equilibrium, that is output exactly meets needs. Any solution $\vx$ that satisfies (\ref{eq:closed}) is called a \emph{steady state} solution.

\begin{pactivity} Is there a steady state solution for the closed system of Agriculture, Manufacturing, and Households? If so, find the general steady state solution. If no, explain why.


\end{pactivity}

So far, we considered the case where the economic system was \emph{closed}. This means that the industries that were part of the system sold products only to each other. However, if we want to represent the demand from other countries, from households, capital building, etc., we need an \emph{open model}. In an article in the \emph{Scientific American} Leontief organized the 1958 American economy into 81 sectors. The production of each of these sectors relied on production from the all of the sectors. Here we present a small sample from Leontief's 81 sectors, using Petroleum, Textiles, Transportation, and Chemicals as our sectors of the economy. Leontief's model assumed that the production of 1 unit of output of
\begin{itemize}
\item petroleum requires 0.1 unit of petroleum, 0.2 units of transportation, and 0.4 units of chemicals;
\item textiles requires 0.4 units of petroleum, 0.1 unit of textiles, 0.15 units of transportation, and 0.3 units of chemicals;
\item transportation requires 0.6 units of petroleum, 0.1 unit of transportation, and 0.25 units of chemicals;
\item chemicals requires 0.2 units of petroleum, 0.1 unit of textiles, 0.3 units of transportation, and 0.2 units of chemicals.
\end{itemize}
A summary of this information is in Table~\ref{T:FourSectors}. Assume the units are measured in dollars.
\begin{table}[h]
\begin{center}
\begin{tabular}{l | c c c c } \\
into\verb+\+from    &Petroleum      & Textiles  &Transportation &Chemicals \\ \hline
Petroleum           &0.10           &0.00       &0.20           &0.40    \\
Textiles            &0.40           &0.10       &0.15           &0.30    \\
Transportation      &0.60           &0.00       &0.10           &0.25    \\
Chemicals           &0.20           &0.10       &0.30           &0.20    \\
\end{tabular}
\caption{Summary of four sector economy.}
\label{T:FourSectors}
\end{center}
\end{table}

In the open model, there is another part of the economy, called the \emph{open sector}, that does not produce goods or services but only consumes them. If this sector (think end consumers, for example) demands/consumes $d_1$ units of Petroleum, $d_2$ units of Textiles, $d_3$ units of Transportation, and $d_4$ units of Chemicals, we put this into a \emph{final demand vector} $\vd = \left[ \begin{array}{c} d_1 \\d_2 \\ d_3 \\ d_4 \end{array}\right]$.

An economist would want to find the production level where the demand from the good/service producing sectors of the economy plus the final demand from the open sector exactly matches the output in each of the sectors. Let $x_1$ represent the number of units of petroleum output, $x_2$ the number of units of textiles output, $x_3$ the number of units of transportation output, and $x_4$ the number of units of chemical output during any time period. Then the production vector is $\vx = \left[\begin{array}{c} x_1 \\ x_2 \\ x_3 \\x_4\end{array}\right]$. So an economist wants to find the production vectors $\vx$ such that
\begin{alignat*}{6}
{0.10}x_1		&{}{}	&{}				&{}+{}	&{0.20}x_3 &{}+{}	&{0.40}x_4 &{}+{}	&{}d_1	&= x_1&{}   \\
{0.40}x_1		&{}+{}	&{0.10}x_2		&{}+{}	&{0.15}x_3 &{}+{}	&{0.30}x_4 &{}+{}	&{}d_2	&= x_2&{}   \\
{0.60}x_1		&{}{}	&{}				&{}+{}	&{0.10}x_3 &{}+{}	&{0.25}x_4 &{}+{}	&{}d_3	&= x_3&{}   \\
{0.20}x_1		&{}+{}	&{0.10}x_2		&{}+{}	&{0.30}x_3 &{}+{}	&{0.20}x_4 &{}+{}	&{}d_4	&= x_4&{,}   \\
\end{alignat*}
where $\vd = \left[ \begin{array}{c} d_1 \\d_2 \\ d_3 \\ d_4 \end{array}\right]$ is the demand vector from the open market.  The matrix 
\[E = \left[ \begin{array}{cccc} 0.10 &0.00 &0.20 &0.40    \\ 0.40  &0.10 &0.15 & 0.30    \\ 0.60 &0.00 &0.10 &0.25 \\  0.20 &0.10 &0.30 &0.20\end{array} \right]\]
derived from Table~\ref{T:FourSectors}, is called the \emph{exchange} matrix.



\begin{pactivity} ~
	\ba
	\item Suppose the final demand vector in our four sector economy is  $\left[ \begin{array}{c} 500 \\200 \\ 400 \\ 100 \end{array}\right]$. Find the production levels that satisfy our system.


\item Does this economy defined by the exchange matrix $E$ have production levels that exactly meet internal and external demands regardless of the external demands? That is, does the system of equations
\begin{alignat*}{6}
{0.10}x_1		&{}{}	&{}				&{}+{}	&{0.20}x_3 &{}+{}	&{0.40}x_4 &{}+{}	&{}d_1	&= x_1&{}   \\
{0.40}x_1		&{}+{}	&{0.10}x_2		&{}+{}	&{0.15}x_3 &{}+{}	&{0.30}x_4 &{}+{}	&{}d_2	&= x_2&{}   \\
{0.60}x_1		&{}{}	&{}				&{}+{}	&{0.10}x_3 &{}+{}	&{0.25}x_4 &{}+{}	&{}d_3	&= x_3&{}   \\
{0.20}x_1		&{}+{}	&{0.10}x_2		&{}+{}	&{0.30}x_3 &{}+{}	&{0.20}x_4 &{}+{}	&{}d_4	&= x_4&{}   \\
\end{alignat*}
have a solution regardless of the values of $d_1$, $d_2$, $d_3$, and $d_4$? Explain.


	\ea
	
\end{pactivity}



