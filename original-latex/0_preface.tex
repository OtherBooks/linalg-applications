\achapter{0}{Preface} 

\csection{A Free and Open-Source Linear Algebra Text} 
 
Mathematics is for everyone -- whether as a gateway to other fields or as background for higher level mathematics. With linear algebra gaining importance in many applications, we feel that access to or the cost of a textbook should not stand in the way of a successful experience in learning linear algebra. Therefore, we made our textbook available to everyone for free download for their own non-commercial use. We especially encourage its use in linear algebra classrooms for instructors who are looking for an inquiry-based textbook or a supplemental resource to accompany their course. If an instructor would like to make changes to any of the files to better suit your students' needs, we offer source files for the text by making a request to the authors.

This work is licensed under the Creative Commons Attribution-NonCommercial-ShareAlike 4.0 International License.  The graphic 
\begin{center}
\includegraphics{CClicense.eps}
\end{center}
that appears throughout the text shows that the work is licensed with the Creative Commons, that the work may be used for free by any party so long as attribution is given to the author(s), that the work and its derivatives are used in the spirit of ``share and share alike,'' and that no party may sell this work or any of its derivatives for profit.  Full details may be found by visiting
\begin{center}
\href{http://creativecommons.org/licenses/by-nc-sa/3.0/}{\texttt{http://creativecommons.org/licenses/by-nc-sa/3.0/}}
\end{center} 
or sending a letter to Creative Commons, 444 Castro Street, Suite 900, Mountain View, California, 94041, USA. 

\csection{Motivation}

The material in this text was developed with a strong focus on linear algebra in $\R^n$ for the first two-thirds of the course, along with an emphasis on applications. We believe that the main ideas of linear algebra are more accessible to students if they are introduced in the more familiar context of vectors in $\R^n$ rather than in abstract vector spaces. For this reason, vector spaces do no arise until late in the text. 

\csection{Goals}

 \emph{An Inquiry-Based Introduction to Linear Algebra and Applications} provides a novel inquiry-based learning approach to linear algebra, as well as incorporating aspects of an inverted classroom. The impetus for this book lies in our approach to teaching linear algebra. We place an emphasis on active learning and on developing students' intuition through their investigation of examples. For us, active learning involves students -- they are DOING something instead of being passive learners. What students are doing when they are actively learning might include discovering, processing, discussing, applying information, writing intensive assignments, engaging in common intellectual in-class experiences or collaborative assignments and projects. Although it is difficult to capture the essence of active learning in a textbook, this book is our attempt to do just that.
 
Additionally, linear algebra is one of the most applicable branches of mathematics to other disciplines. In fact, the Linear Algebra Curriculum Study Group (LACSG) (formed in 1990 to ``initiate substantial and sustained national interest in improving the undergraduate linear algebra curriculum" and funded by the National Science Foundation) made recommendations for the linear algebra curriculum, one of which was making linear algebra a matrix-oriented course that includes applications to give an indication of the ``pervasive use" of linear algebra in other disciplines.\footnote{David Carlson, Charles R. Johnson, David Lay, Duane Porter. ``The Linear Algebra Curriculum Study Group Recommendations for the First Course in Linear Algebra.'' \emph{The College Mathematics Journal}, Vol. 24, No. 1, 1993, 41-46.}  Also, the most recent Committee on the Undergraduate Program in Mathematics (CUPM) report \footnote{\emph{2015 CUPM Curriculum Guide to Majors in the Mathematical Sciences}, Carol S. Schumacher and Martha J. Siegel, Co-Chairs, Paul Zorn, Editor.} to the Mathematical Association of America recommends ``every Linear Algebra course should incorporate interesting applications, both to highlight the broad usefulness of linear algebra and to help students see the role of the theory in the subject as it is applied." An important component of this text is its inclusion of significant ``real-life" applications of linear algebra to help students see that linear algebra is widely applicable in many disciplines. 
 %\url{https://www.maa.org/sites/default/files/CUPM%20Guide.pdf}
Our goals for these materials are several.
\begin{itemize}
\item To carefully introduce the ideas behind the definitions and theorems to help students develop intuition and understand the logic behind them.
\item To help students understand that mathematics is not done as it is often presented. We expect students to experiment through examples, make conjectures, and then refine their conjectures. We believe it is important for students to learn that definitions and theorems don't pop up completely formed in the minds of most mathematicians, but are the result of much thought and work.
\item To help students develop their communication skills in mathematics. We expect our students to read and complete activities before class and come prepared with questions. While in class, students work to discover many concepts on their own through guided activities.  Of course, students also individually write solutions to exercises on which they receive significant feedback. Communication skills are essential in any discipline and we place a heavy focus on their development.
\item To have students actively involved in each of these items through in-class and out-of-class activities, in-class presentations (this is of course up to the instructor), and problem sets.
\item To expose students to significant real-life applications of linear algebra.
\end{itemize}

\csection{Layout}

This text is formatted into sections, each of which contains preview activities, in-class activities, worked examples, and exercises. Most sections conclude with an application project -- an application of the material in the section. The various types of activities serve different purposes.
\begin{itemize}
\item Preview activities are designed for students to complete before class to motivate the upcoming topic and prepare them with the background and information they need for the class activities and discussion.
\item We generally use the regular activities to engage students during class in critical thinking experiences. These activities are used to provide motivation for the material, opportunities for students to develop course material on their own, or examples to help reinforce the meanings of definitions or theorems. The ultimate goal is to help students develop their intuition for and understanding of linear algebra concepts. 
\item Worked examples are included in each section. Part of the philosophy of this text is that students will develop the tools and understanding they need to work homework assignments through the preview, in-class activities, and class discussions. But some students express a desire for fully-worked examples in the text to reference during homework. In order to preserve the flow of material within a section, we include worked examples at the end of each section.  
\item The suggestion by the Linear Algebra Curriculum Study Group (LACSG) that, ``Mathematics departments should seriously consider making their first course in linear algebra a matrix-oriented course." is followed in this text. They suggest that this approach ``implies less emphasis on abstraction and more emphasis on problem solving and motivating applications." and that ``Some applications of linear algebra should be included to give an indication of the pervasive use of linear algebra in many client disciplines. Such applications necessarily will be limited by the need to minimize technical jargon and informa�tion from outside the course. Students should see the course as one of the most potentially useful mathematics courses they will take as an undergraduate." The focus of this text for the first two semesters of linear algebra (Sections 1 through 27) is on $\R^n$, and applications of linear algebra in real spaces. At our institution, these sections comprise a two-semester entry-level sequence into our major, suitable for freshmen without a calculus prerequisite. Abstract vector spaces do not appear until Section 28, which we consider to be the beginning of a third semester of linear algebra, a junior or senior level mathematics course. Also, the 2015 Committee on the Undergraduate Program in Mathematics (CUPM) report to the Mathematical Association of America recommends that ``Every Linear Algebra course should incorporate interesting applications, both to highlight the broad usefulness of linear algebra and to help students see the role of the theory in the subject as it is applied.  Attractive applications may also entice students majoring in other disciplines to choose a minor or additional major in mathematics." All but two sections in this text include a substantial application. Each section begins with a short description of an application that uses the material from the section, then concludes with a project that develops the application in more detail. (The two sections that do not include projects are sections that are essentially long proofs -- one section that contains formal proofs of the equivalences of the different parts of the Invertible Matrix Theorem, and the other contains algebraic proofs of the properties of the determinant.) The projects are independent of the material in the text -- the text can be used without the applications. The applications are written in such a way that they could also be used with other textbooks. The projects are written following an inquiry-based style similar to the text, with important parts of the applications developed through activities. So the projects can be assigned outside of class as independent work for students. Several of the projects are accompanied by GeoGebra applets or Sage worksheets which are designed to help the students better understand the applications. 
\item Each investigation contains a collection of exercises. The exercises occur at a variety of levels of difficulty and most force students to extend their knowledge in different ways. While there are some standard, classic problems that are included in the exercises, many problems are open ended and expect a student to develop and then verify conjectures.
\end{itemize}

\csection{Acknowledgements} 

The many beautiful images that are in this text were created by our colleague David Austin, to whom we owe a debt of gratitude. Our institution, Grand Valley State University, supported our development of these materials early on through an internal university grant. Finally, we wish to thank our GVSU colleagues for their support and for being role models in the open source textbook movement. 

\csection{To the Student}

The inquiry-based format of this book means that you can be in charge of your own learning. The guidance of your instructor and support of your fellow classmates can help you develop your understanding of the topic. Your learning will benefit best if you engage in the material by completing all preview and in-class activities in order to fully develop your own intuition and understanding of the material, which can be achieved only by reflecting on the ideas and concepts and communicating your reflections with others. Don't be afraid to ask questions or to be wrong at times, for this is how people learn. Good luck! We will be happy to hear your comments about the book.



